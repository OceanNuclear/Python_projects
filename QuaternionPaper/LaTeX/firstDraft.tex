\documentclass[10pt,a4paper,twocolumn]{article}
\usepackage[utf8]{inputenc}
\usepackage{amsmath}
\usepackage{amsfonts}
\usepackage{amssymb}

\author{
  Rodriguez, David Gonzalez\\
  \texttt{d.gonzalezrodriguez@bham.ac.uk}
  \and
  Wong, Hoi Yeung\\
  \texttt{OceanWongUK@gmail.com}
}

\begin{document}
\maketitle
\begin{abstract}
%Something for abstract?
\end{abstract}

\section{Introduction}
%Importance of quantifying the GOS [vs strain]

\section{Constitutive model}
%L -> F= F^p F^e
%L= Le Lp, asym(L) = D

\section{Mathematical treatment}
	\subsection{Orientation averaging methods}
	%plot several grains (pick 10 out of 125)
	%Make a list of methods here*
		%explain the methods concisely;
		%put the rest in annex/appendix.
	\subsection{Grain Orientation Scatter}
	%Give GOS formula

\section{Finite Element Model (FEM)}
%How is the model created
%Boundary conditions applied

\section{Results}
	\subsection{Effect of grain size on GOS}
	%Plot grain size distribution
	%GOS vs volume
	%Explain these two results

	\subsection{Effect of strain on GOS}
		\subsubsection{macroscopic strain}
		%Plot GOS vs ε (macroscopic strain)
			%plot lower SD vs higher SD as well
		\subsubsection{straind direction}
		%Weighted GOS(+GOS?) vs volume
			%use the heat-map plot for plotting GOS on IPF (/contour)
	%Explain these two results
	
	\subsection{Effect of constraints}
	%GOS vs $\theta$ plot
	%KAM vs x on GB
	%Other plots?
	%explanation

\section{Conclusion}
%Something

\section{Refernces}

\begin{thebibliography}
%Insert references here.
\end{thebibliography}
\end{document}