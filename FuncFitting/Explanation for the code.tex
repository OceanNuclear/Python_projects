\documentclass[a4paper, 12pt]{article}
\usepackage{graphicx}
\begin{document}

The method below is "general" enough to be applied onto any function one wishes to fit their data with.

Suppose we have n data points ($x_i$, $y_i$) each with error in y of $dy_i$ for $0 \leq i \leq n$, to be fitted to a function $y_{calc,i}(x_i)$, then the residual of the function $= y_{calc,i} - y_i$

To fit the function, we aim to minimize the following:

\[S = \sum_i (\frac{y_{calc,i} - y_i}{dy_i})^2\]
Which is the weighted "version" of the residual.

If the the fitting variables are a and b, i.e. $y_{calc}$ = $y_{calc}(x,a,b)$ then $S$ must be quadratic with respect to variation in $y_{calc}$, and therefore reasonably linearly wrt. variation in $a$, $b$.

To find the most suitable value of a and b is to find the minimum value of S wrt. a, b.

\[\frac{\partial S}{\partial a} = 0\] 
\[\frac{\partial S}{\partial b} = 0\] 

For a,

\[\frac{\partial}{\partial a}\sum_i (\frac{y_{calc,i} - y_i}{dy_i})^2 = 0\]

Simplifying, and removing the i subscripts to minimize the visual complexity of the equation,

\[\sum_i (\frac{y_{calc} - y}{dy})
(\frac{\partial y_{calc}}{\partial a}) = 0\]

The equation above can be extended to b (or c, d, etc. if there are more than 2 variables) if we replace all a in the equation with b.

($\frac{\partial y_{calc}}{\partial a}$ is represented as 
\texttt{dDevda} in the code.)
\end{document}